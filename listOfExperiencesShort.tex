\textbf{{Brookhaven National Laboratory $\>$$\>$$\>$$\>$Computer Scientist$\>$$\>$$\>$$\>$May 2019 - present}}
\begin{itemize}
    \item Working on locality-aware loop scheduling strategies, and generally user-defined loop schedules, in the context of MPI+OpenMP applications
    \item Developing novel performance tuning techniques for MPI+OpenMP applications for emerging supercomputers, with a focus on new and challenging computational simulations and next-generation computer architectures.
     \item   Serving as Program Manager for DoE Exascale Computing Project’s project SOLLVE and asBrookhaven National Laboratory’s representative in the OpenMP Architecture Review Board.
\end{itemize}

\dates{Jun. '18 - Apr. '19}
\location{Champaign, Illinois}
\title{Software Developer}
\employer{Charmworks, Inc.} 
\textbf{Charmworks, Inc. $\>$$\>$$\>$$\>$Software Developer$\>$$\>$$\>$$\>$Jun. '18 - Apr. '19}
%\begin{position}
\vspace{0.0in}
\begin{itemize}
\item Worked on a performance optimization strategies that synergize loop scheduling done within-node with  load balancing done across-node. 
\item Involved in contributing sections on loop scheduling in the OpenMP Specification (specification at openmp.org).
\item Worked on a startup proposal for loop scheduling for applications for desktop computers (as opposed to HPC Applications) 
%\item Assisting with slides for pitch and marketing of Charm++ software, and providing feedback for tutorials on Charm++.
\item Integrated a shared memory library for sophisticated loop scheduling strategies, including some based on my dissertation, into the current version of Charm++.
%item Comparing performance of a loop scheduling strategy available in the integrated shared memory library with the performance of the corresponding loop scheduling strategy available in LLVM’s OpenMP library.
\end{itemize}
%\end{position}

\textbf{University of Southern California$\>$$\>$$\>$$\>$Computer Scientist$\>$$\>$$\>$$\>$Dec. '16 - Jun '18}
\vspace*{-0.0in} 
\begin{itemize}
\item Worked with postdoc from LLNL on a proposal to study techniques that combine loop scheduling and load balancing to improve
performance of scientific applications.
\item Worked with OpenMP Language Committee to support user-defined loop schedules in OpenMP.
\item Translated an x-ray tomography code written in Matlab code to C code and then parallelizing it to run on a supercomputer having nodes with GPGPUs. 
\item Worked on modifications to LLVM compiler to support new OpenMP loop schedules. 
% \item Worked on ensuring external network infrastructure to support transfer of application code's input data files were adequate for an application code's efficient execution using the Globus Toolkit.
%\item \small Managing a git repository for a team working on
%performance optimizations of the application program.
\item Worked in team to manage computational performance aspects of running an application program involving Fast Fourier Transformation and image reconstruction algorithms. 
%\item \small Doing optimizations for MPI+CUDA application code involving low-overhead loop scheduling and loop optimizations such as loop unrolling. 
%\item \small Working on transformations in LLVM. 
\end{itemize}

%TODO: adaptive VS hybrid VS ... 
\textbf{Charmworks, Inc.$\>$$\>$$\>$$\>$Developer$\>$$\>$$\>$$\>$Jan. '16 - Nov. '16}
\vspace*{-0.0in}
\begin{itemize}
\item Implemented mixed static/dynamic loop scheduling
strategies within Charm++'s thread scheduling library.
%TODO: consider adding 'including in cloud environments' the end of
%the sentence. 
%TODO: make paragraph 
\item Helped to improve portability of Charm++ to a variety of platforms. 
\item Assisted with business aspects of a high-tech startup. 
\end{itemize} 

\textbf{ University of Illinois$\>$$\>$$\>$$\>$Postdoctoral Associate$\>$$\>$$\>$$\>$Jul. '15 – Dec. '15}
\vspace*{-0.0in}
\begin{itemize} 
\item Developed library that allows application programmers to use strategies from dissertation.
\item Adapted a plasma physics application code to work on a
GPGPU processor and Intel Xeon Phi.
\item Incorporated over-decomposition and locality-aware scheduling into strategies from dissertation.
\end{itemize}

\textbf{Lawrence Livermore Nat’l Lab$\>$$\>$$\>$$\>$Lawrence Scholar$\>$$\>$$\>$$\>$Feb. '12 – Jun. '14}
\vspace*{-0.0in}
\begin{itemize} 
\item Measured MPI communication delays for micro-benchmarks codes run on supercomputers and worked to find tools to measure dequeue overheads of OpenMP loop schedulers.
\item Created a software system for automated performance optimization and application programmer usability of low-overhead hybrid scheduling
strategies.
\item Developed a ROSE-based custom compiler for automatically transforming MPI+OpenMP applications to use low-overhead scheduling
techniques and runtime.
\item Assessed further opportunities for performance improvement of low-overhead schedulers, including improvement of spatial locality
of low-overhead schedulers.
\end{itemize}

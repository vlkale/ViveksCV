
\textbf{{Sandia National Laboratories $\>$$\>$$\>$  Principal Member of Technical Staff II $\>$$\>$$\>$Jul 2024 - present}} 
\begin{itemize}
\item Maintainer of HPC Tools and Runtime Interoperability at Sandia Labs, including liason for LLNL Performance Tools and maintainer for Kokkos Tools. 
\item Sandia Rep and contributor to OpenMP specification and MPI forum as Sandia Representative. 
\end{itemize} 
\dates{August 2022 - July 2024}
\location{Livermore, California, USA}
\title{Senior Member of Technical Staff}
\employer{Sandia National Laboratories}

\textbf{{Sandia National Laboratories $\>$$\>$$\>$ Senior Member of Technical Staff $\>$$\>$$\>$Aug 2022 - Jul 2024}}
\begin{itemize}
	\item Same as above.
%\item Developing and testing features in the US DoE's LLVM's OpenMP implementation. 
%\item Contributing to OpenMP 6.0 Specification, specifically on topics of affinity, loop transformations, accelerators and tasking.
%\item Prototyping tunable locality-aware loop scheduling strategy features for OpenMP, and generally user-defined loop schedules, for LLVM's OpenMP implementation.  

%\item Owner of Kokkos Software Ecosystem's Kokkos Tools, which provides profiling and debugging capabilities for Kokkos programs (for performance portable parallel programs) as well as sophisticated auto-tuning and performance analysis capabilities.
%\item Contributor to the DOE ASCR Xstack project on automated test generation for parallel programs via LLVM. Developing a source-to-source translator via the ROSE compiler plugin for the LLVM's clangASTRewriter to translate a Kokkos program to a Kokkos Model (simplified version of Kokkos) program for analysis by LLVM's Klee symbolic execution library.
\end{itemize}

\dates{May 2019 - August 2022}
\location{Upton, New York, USA}
\title{Computational Scientist} 
\employer{Brookhaven National Laboratory} 
\textbf{{Brookhaven National Laboratory $\>$$\>$$\>$$\>$Computational Scientist$\>$$\>$$\>$$\>$May 2019 - Aug 2022}}
\begin{itemize}
  % \item Contributed to developing an LLVM OpenMP implementation, specifically the OpenMP implementation's compiler and its runtime, targetted for Department of Energy's upcoming Exascale Supercomputer platforms. 
   \item Designed and implemented LLVM's OpenMP user-defined and task-to-multiGPU scheduling strategies to improve within-node load balancing of applications running on supercomputers having multiple GPUs per node.
   \item Developed benchmarks and evaluating OpenMP implementations on Exascale Supercomputers.
   \item Represented Brookhaven National Laboratory in the OpenMP Architecture Review Board. 
\end{itemize}

\dates{June 2018 - April 2019}
\location{Champaign, Illinois, USA}
\title{Software Developer}
\employer{Charmworks, Inc.}
\textbf{Charmworks, Inc. $\>$$\>$$\>$$\>$Software Developer$\>$$\>$$\>$$\>$Jun 2018 - Apr 2019}
%\begin{position}
\vspace{0.0in}
\begin{itemize}
%\item Collaborated with Lawrence Livermore National Lab on a proposal for a synergistic loop scheduling and load balancing strategy. 
%\item Worked on making User-defined Loop Scheduling portable across different parallel programming library, done with Oak Ridge National Lab through DoE Exascale Computing Program. 
%\item Added examples of loop scheduling in OpenMP in the Examples section of OpenMP Specification.
%\item Worked on a NSF startup SBIR proposal for loop scheduling for desktop computers. 
%\item Collaborated on developing a proposal to add an OpenMP User-defined Schedule to the OpenMP specification based on an OpenMPCon 2017 paper, presenting a proposal at the OpenMP F2F in Santa Clara and the upcoming F2F in Toronto.
\item Did research and development for User-defined Loop Schedules in OpenMP.
%\item Assisted with slides for pitch and marketing of Charm++ software, and providing feedback for tutorials on Charm++.
\item Integrated a shared memory library for sophisticated loop scheduling strategies, including some based on my dissertation, into the current version of Charm++. 
%item Comparing performance of a loop scheduling strategy available in the integrated shared memory library with the performance of the corresponding loop scheduling strategy available in LLVM’s OpenMP library.
\end{itemize}

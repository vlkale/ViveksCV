\documentclass[11pt]{article}
\usepackage[margin=1in]{geometry}
\usepackage{hyperref}
\usepackage[T1]{fontenc}
\usepackage[utf8]{inputenc}
\usepackage{lmodern}
\begin{document}
\begin{flushright}
\textbf{Vivek L. Kale}\\
\href{tel:+16507339327}{(650) 733-9327} \\ \href{mailto:vivek.lkale@gmail.com}{vivek.lkale@gmail.com} \\ \href{https://www.linkedin.com/in/vlkale}{linkedin.com/in/vlkale} \\ \href{https://github.com/vlkale}{github.com/vlkale} \\ \href{https://vlkale.github.io}{vlkale.github.io}
\end{flushright}
\vspace{0.25in}
\noindent December 12, 2025\\
Hiring Manager\\
[Company Name]\\
[City, State]

\vspace{0.15in}
\noindent Dear Hiring Manager,

\vspace{0.1in}
I specialize in AI GPU techniques and kernel performance engineering: optimizing CUDA/HIP/ROCm kernels, MLIR- and LLVM-based transformations, memory hierarchy tuning (HBM/NVLink/PCIe), and warp-level efficiency (occupancy, divergence, latency hiding). I’ve delivered measurable speedups across transformer and inference workloads via fusion, vectorization, and shared-memory tiling, backed by rigorous profiling with Nsight, VTune, Perfetto, TAU, HPCToolkit, and OMPT/NVTX instrumentation.

I partner with model and systems teams to diagnose bottlenecks end-to-end (data movement, kernel launch, runtime scheduling) and implement reproducible performance improvements with correctness checks, CI perf gates, and cross-vendor benchmarking on NVIDIA, AMD, and Intel GPUs. I’m excited to bring hands-on kernel optimization, compiler tooling, and performance methodology to strengthen your AI GPU stack.

\vspace{0.15in}
Sincerely,\\
Vivek L. Kale
\end{document}